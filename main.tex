\documentclass{article}

\usepackage{amsmath, amsthm, amssymb, amsfonts}
\usepackage{thmtools}
\usepackage{graphicx}
\usepackage{setspace}
\usepackage{geometry}
\usepackage{float}
\usepackage{hyperref}
\usepackage[utf8]{inputenc}
\usepackage[english]{babel}
\usepackage{framed}
\usepackage[dvipsnames]{xcolor}
\usepackage{tcolorbox}

\colorlet{LightGray}{White!90!Periwinkle}
\colorlet{LightOrange}{Orange!15}
\colorlet{LightGreen}{Green!15}
\colorlet{LightBlue}{Cyan!15}

\newcommand{\HRule}[1]{\rule{\linewidth}{#1}}

\declaretheoremstyle[name=Theorem,]{thmsty}
\declaretheorem[style=thmsty,numberwithin=section]{theorem}
\tcolorboxenvironment{theorem}{colback=LightGray}

\declaretheoremstyle[name=Proposition,]{prosty}
\declaretheorem[style=prosty,numberlike=theorem]{proposition}
\tcolorboxenvironment{proposition}{colback=LightOrange}

\declaretheoremstyle[name=Principle,]{prcpsty}
\declaretheorem[style=prcpsty,numberlike=theorem]{principle}
\tcolorboxenvironment{principle}{colback=LightGreen}

\declaretheoremstyle[name=Summary,]{sumsty}
\declaretheorem[style=sumsty, numbered=no]{summary}
\tcolorboxenvironment{summary}{standard jigsaw, opacityback=0}

\theoremstyle{remark}
\newtheorem*{remark}{Remark}

\declaretheoremstyle[
    name=Definition,
    headfont=\bfseries, % bold heading
    notebraces={(}{)},  % format for the heading
    headpunct={:},      % punctuation after the heading
    postheadspace=1em,  % space after the heading
    spaceabove=\baselineskip,
    spacebelow=\baselineskip,
]{defsty}
\declaretheorem[style=defsty,numberlike=theorem]{definition}
\tcolorboxenvironment{definition}{colback=LightBlue}

\declaretheoremstyle[
    name=Example,
    headfont=\itshape, % italicized heading
    notebraces={(}{)},  % format for the heading
    headpunct={:},      % punctuation after the heading
    postheadspace=1em,  % space after the heading
    spaceabove=\baselineskip,
    spacebelow=\baselineskip,
]{exsty}
\declaretheorem[style=exsty,numberwithin=theorem]{example}


\setstretch{1.2}
\geometry{
    textheight=9in,
    textwidth=5.5in,
    top=1in,
    headheight=12pt,
    headsep=25pt,
    footskip=30pt
}

% ------------------------------------------------------------------------------

\begin{document}

% ------------------------------------------------------------------------------
% Cover Page and ToC
% ------------------------------------------------------------------------------

\title{ \normalsize \textsc{}
		\\ [2.0cm]
		\HRule{1.5pt} \\
		\LARGE \textbf{\uppercase{Optimal Control}
		\HRule{2.0pt} \\ [0.6cm] \LARGE{MAE 546 : Fall 2024} \vspace*{10\baselineskip}}
		}
\date{}
\author{\textbf{Nathaniel Chen} \\ 
		Ryne Beeson \\
		Andlinger 017 \\
		9:30-10:50 AM}

\maketitle
\newpage

\tableofcontents
\newpage

% ------------------------------------------------------------------------------

\section{Optimal Control Introduction}

\subsection{Deterministic Finite-Dimensional Continuous-Time Problem}

\begin{equation}
    \inf_{u\in\mathcal{U}} J(u;t_0,t_f,X_0) \equiv K(t_f,X_f) + \int_{t_0}^{t_f} L(s,X_s,u_s)\,ds
\end{equation}
\begin{equation}
    \,dX_t \equiv f(t,X_t, u_t)\,dt\quad X_0 \in \mathbb{R}^m
\end{equation}
\begin{equation}
    \psi(t,X_t,u_t) = 0 \in \mathbb{R}^l,\quad \forall t \in [t_0,t_f]
\end{equation}
\begin{equation}
    \phi(t_f,X_f,u_t) \le 0 \in \mathbb{R}^k, \quad \forall t \in [t_0,t_f]
\end{equation}

\subsection{Definitions}

\begin{summary}

    Fundamental definitions
    \begin{enumerate}
        \item Metric Space: $(M,d)$
        \item Inner Product Induced Metric: $(M, \langle \cdot, \cdot \rangle)$
        \item Topology: $\mathcal{T} \equiv (A_i)$
        \item Open \& Closed Sets: $A, A^c$
        \item Open \& Closed Balls: $B(x,\epsilon;d), \bar{B}(x,\epsilon;d)$
        \item Metric Topology: $\mathcal{T}(M)$
        \item Set Closure: $\bar{A}$
        \item Set Interior: $A^\circ$
        \item Open Neighborhood: $A\subseteq M$
    \end{enumerate}
\end{summary}

\begin{definition}[Metric Space]

    Defining
    \begin{itemize}
        \item $(M,d)$: a metric space
        \item $M$: a set with topology induced by $d$
        \item $d: M \times M -> [0, \infty)$
    \end{itemize}
    Then
    \begin{enumerate}
        \item d(x,y) = d(y,x) \forall x,y \in M \hfill\text{Symmetric}
        \item d(x,x) = 0 \forall x \in M
        \item d(x,y) > 0, \forall x,y \in M, x \neq y \hfill\text{Non-Negative}
        \item d(x,y) \leq d(x,z) + d(z,y) \forall x,y,z \in M\hfill{Triangle Inequality}
    \end{enumerate}

\end{definition}

\begin{definition}[Inner Product Induced Metric]

    Defining
    \begin{itemize}
        \item $(M, \langle \cdot, \cdot \rangle)$: an inner product space
        \item $M$: a vector space
        \item $\langle \cdot, \cdot \rangle$: an inner product
    \end{itemize}
    This induces the metric
    \begin{equation}
        d(x,y) = |x-y| \equiv \langle x-y, x-y \rangle^{1/2}, \quad\forall x,y \in M
    \end{equation}

\end{definition}

\begin{definition}[Topology]

    Defining
    \begin{itemize}
        \item $\mathcal{T} \equiv (A_i)$: a collection of subsets of $M$
    \end{itemize}
    $\mathcal{T}$ forms a topology for $M$ if the following hold
    \begin{enumerate}
        \item $M, \emptyset \in \mathcal{T}$
        \item If $(E_i) \subseteq \mathcal{T}$ is a countable collection
        \item If $(E_i) \subseteq \mathcal{T}$ is a finite collection
    \end{enumerate}
    
\end{definition}

\begin{definition}[Open \& Closed Sets]

    \begin{itemize}
        \item Open Set: elements of a topology (i.e., $A \in \mathcal{T}$)
        \item Closed Set: complement of an open set (i.e., $A^c$)
    \end{itemize}
    
\end{definition}

\begin{definition}[Open \& Closed Balls]

    Defining
    \begin{itemize}
        \item $(M,d)$: a metric space
        \item $\epsilon > 0$: Radius
        \item $x \in M$: Center
    \end{itemize}
    Open and closed balls are defined as
    \begin{enumerate}
        \item Open Ball: $B(x,\epsilon;d) \equiv \{y \in M | d(x,y) < \epsilon\}$
        \item Closed Ball: $\bar{B}(x,\epsilon;d) \equiv \{y \in M | d(x,y) \le \epsilon\}$
    \end{enumerate}
    
\end{definition}

\begin{definition}[Metric Topology]
    
    Given
    \begin{itemize}
        \item $(M,d)$: a metric space
    \end{itemize}
    The metric can induce a topology by considering a collection of open balls.

    \begin{example}[Borel Topology]

        \begin{itemize}
            \item standard topology on $\mathbb{R}^m$
            \item all open balls centered at rational numbers $\mathbb{Q}$
            \item radius is positive rational
        \end{itemize}
    \end{example}

\end{definition}

\begin{definition}[Set Closure]

    Given
    \begin{itemize}
        \item $A \subseteq M$: a subset of a metric space
        \item ${D_i}$: collection of all closed sets that contain $A$
        \item $\bar{A} \supseteq A$: the closure of $A$
    \end{itemize}
    Closure is defined as
    \begin{equation}
        \bar{A} \equiv \bigcap_{i} D_i
    \end{equation}

    \begin{remark}
        In Borel topology, isolated points are closed
    \end{remark}
\end{definition}

\begin{definition}[Set Interior]

    Given
    \begin{itemize}
        \item $A \subseteq M$: a subset of a metric space
        \item ${E_i}$: collection of all open sets that contain $A$
        \item $A^\circ \subseteq A$: the interior of $A$
    \end{itemize}
    The interior is defined as
    \begin{equation}
        A^\circ \equiv \bigcup_{i} E_i
    \end{equation}
    
\end{definition}

\begin{definition}[Open Neighborhood]

    Given
    \begin{itemize}
        \item $x \in M$: a point in a metric space
        \item $\mathcal{T}(M)$: the topology of $M$
    \end{itemize}
    $A \subseteq M$ is an open neighborhood of $x$ if
    \begin{itemize}
        \item $x \in A$
        \item $A \in \mathcal{T}(M)$
    \end{itemize}
    
    \begin{remark}
        The neighborhood is implied to be small, with motivation from metric topology implying that $A$ looks like a small open ball centered at $x$.
    \end{remark}
\end{definition}
\section{Parameter Optimization Conditions}

\subsection{Defining Optimality}

\begin{summary}

Defining local and global minimum on a metric space for a cost function

\begin{enumerate}
    \item Local minimum
    \item Global minimum
    \item Local minimum; $(\mathbb{R},d$): Local minimum in standard 1D metric space
    \item Global minimum; $(\mathbb{R},d)$: Global minimum in standard 1D metric space
    \item Extremum
\end{enumerate}

\end{summary}

\subsection{Unconstrained Smooth Parameter Optimization}

\begin{summary}

Deriving optimiality conditions in Euclidean space $\mathbb{R}^n$

Definitions:
\begin{enumerate}
    \item Continuously Bounded Differentiable Function
    \item Compact Set
    \item Stationary Point
    \item Unit Sphere
    \item Hessian Matrix
\end{enumerate}

Theorems:
\begin{enumerate}
    \item Heine-Borel Property
    \item Weierstrass Extreme Value
    \item First Order Necessary Condition; $(\mathbb{R},d)$
    \item Taylor's Formula; Using Lagrange Form of the Mean-Value of the Remainder
    \item Second Order Necessary Condition; $(\mathbb{R},d)$
    \item Second Order Sufficient Condition; $(\mathbb{R},d)$
    \item First Order Necessary Condition; $(\mathbb{R}^n,d)$
\end{enumerate}

\end{summary}

\newpage

% ------------------------------------------------------------------------------
% Reference and Cited Works
% ------------------------------------------------------------------------------

\bibliographystyle{IEEEtran}
\bibliography{References.bib}

% ------------------------------------------------------------------------------

\end{document}
