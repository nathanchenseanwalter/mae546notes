\section{Parameter Optimization Conditions}

\subsection{Defining Optimality}

\begin{summary}

Defining local and global minimum on a metric space for a cost function

\begin{enumerate}
    \item Local minimum
    \item Global minimum
    \item Local minimum; $(\mathbb{R},d$): Local minimum in standard 1D metric space
    \item Global minimum; $(\mathbb{R},d)$: Global minimum in standard 1D metric space
    \item Extremum
\end{enumerate}

\end{summary}

\begin{definition}[Local Minimum]
    
    Given
    \begin{itemize}
        \item $(M,d)$: a metric space
        \item $\mathcal{T}$: a topology on $M$ induced by the metric $d$
        \item $f: M \rightarrow \mathbb{R}$: a cost function
    \end{itemize}

    $x^* \in M$ is a local minimum of $f$ if $\exists$ neighborhood $A$ of $x^*$ such that
    \begin{equation}
        f(x^*) \leq f(x)\,\quad \forall x \in A
    \end{equation}

    \begin{remark}
        $x^*$ is a strict local minimum if
        \begin{equation}
            f(x^*) < f(x)\,\quad \forall x \in A\backslash x^*
        \end{equation}
    \end{remark}

\end{definition}

\begin{definition}[Global Minimum]

    Given
    \begin{itemize}
        \item $(M,d)$: a metric space
        \item $\mathcal{T}$: a topology on $M$ induced by the metric $d$
        \item $f: M \rightarrow \mathbb{R}$: a cost function
    \end{itemize}

    $x^* \in M$ is a global minimum of $f$ if
    \begin{equation}
        f(x^*) \leq f(x)\,\quad \forall x \in M
    \end{equation}
    
\end{definition}

\begin{definition}[Local Minimum; $(\mathbb{R}, d)$]

    $x^* \in \mathbb{R}$ is a local minimum of $f: \mathbb{R} \rightarrow \mathbb{R}$ if $\exists \epsilon > 0$ such that
    \begin{equation}
        f(x^*) \leq f(x)\,\quad \forall x \in (x^* - \epsilon, x^* + \epsilon) = B(x^*, \epsilon)
    \end{equation}

    $x^*$ is a strict local minimum if
    \begin{equation}
        f(x^*) < f(x)\,\quad \forall x \in (x^* - \epsilon, x^* + \epsilon)\backslash \{x^*\}
    \end{equation}

\end{definition}

\begin{definition}[Global Minimum; $(\mathbb{R}, d)$]

    $x^* \in \mathbb{R}$ is a global minimum of $f: \mathbb{R} \rightarrow \mathbb{R}$ if
    \begin{equation}
        f(x^*) \leq f(x)\,\quad \forall x \in \mathbb{R}
    \end{equation}
    $x^*$ is a strict global minimum if
    \begin{equation}
        f(x^*) < f(x)\,\quad \forall x \in \mathbb{R}\backslash \{x^*\}
    \end{equation}
    
\end{definition}

\begin{definition}[Extremum]
    
    If $x^*$ is a local minimum or maximum, then $x^*$ is an extremum.

\end{definition}

\subsection{Unconstrained Smooth Parameter Optimization}

\begin{summary}

Deriving first and second order necessary and sufficient optimiality conditions in Euclidean space $\mathbb{R}^n$.

Definitions:
\begin{enumerate}
    \item Continuously Bounded Differentiable Function
    \item Compact Set
    \item Stationary Point
    \item Unit Sphere
    \item Hessian Matrix
\end{enumerate}

Theorems:
\begin{enumerate}
    \item Heine-Borel Property
    \item Weierstrass Extreme Value
    \item First Order Necessary Condition; $(\mathbb{R},d)$
    \item Taylor's Formula; Using Lagrange Form of the Mean-Value of the Remainder
    \item Second Order Necessary Condition; $(\mathbb{R},d)$
    \item Second Order Sufficient Condition; $(\mathbb{R},d)$
    \item First Order Necessary Condition; $(\mathbb{R}^n,d)$
\end{enumerate}

\end{summary}

\begin{definition}[Continuously Bounded Differentiable Function]

    Defining
    \begin{itemize}
        \item $C^k(\Omega;\mathbb{R})$: real-valued $k$-times continuously differentiable functions on the set $\Omega$. So $f\in C^k(\Omega;\mathbb{R})$ is k-times differentiable, and each derivative is continuous. e.g., $\Omega \subseteq \mathbb{R}^n$
        \item $C^k_b(\Omega;\mathbb{R})$: real-valued $k$-times continuously differentiable functions on the set $\Omega$ that are bounded
    \end{itemize}
    
\end{definition}

\begin{definition}[Compact Set]

    Given
    \begin{itemize}
        \item $(M,\mathcal{T})$: a topological space
        \item $\Omega \subseteq M$: a set
    \end{itemize}

    $\Omega$ is compact if
    
    \begin{itemize}
        \item $(E_i)_{i\in I}$: open cover of $\Omega$
        \item $(E_j)_{j\in J}$: finite subcover where $J \subseteq I$
        \item For any $(E_i)_{i\in I}$, there exists $(E_j)_{j\in J}$.
    \end{itemize}
\end{definition}

\begin{theorem}[Heine-Borel Property]

    A metric space has the Heine-Borel property $iff$ every compact set is closed and bounded. In particular, $\mathbb{R}^n$ with the sandard metric has this property.
    
\end{theorem}

\begin{theorem}[Weierstrass Extreme Value]
    
    Let
    \begin{itemize}
        \item $\Omega$: a compact set
        \item $f:\omega\to\mathbb{R}$ a continous function.
    \end{itemize}

    Then $f$ has extremums on $\Omega$.

    \begin{remark}
        If $\Omega$ is compact, then the class of functions is automatically $C^k_b(\Omega;\mathbb{R})$
    \end{remark}

\end{theorem}

\begin{theorem}[First Order Necessary Condition; $(\mathbb{R},D)$]
    
    Given
    \begin{itemize}
        \item $f\in C^1(\Omega;\mathbb{R})$
        \item $x^* \in \Omega^\circ \subseteq \mathbb{R}$
    \end{itemize}

    If $x^*$ is a local minimum for $f$, then
    
    \begin{equation*}
        \pdv{f}{x}_{x^*} = 0
    \end{equation*}

\end{theorem}

\begin{definition}[Stationary Point]
    
    $x^*$ is a stationary point for $f$ if

    \begin{equation}
        \partial_xf|_{x^*} = 0
    \end{equation}
\end{definition}

\begin{theorem}[Taylor's Formula; Using Language Form of the Mean-Value of the Remainder]
    
    Assume
    \begin{itemize}
        \item $f\in C^2_b(\mathbb{R};\mathbb{R})$
        \item $x,x^*$
        \item $\delta x\equiv x-x^*$
    \end{itemize}
    
    there exists a point $y$ between $x$ and $x^*$ such that

    \begin{equation}
        f(x) = f(x^*) + \delta_x f|_{x^*}\delta x + \frac{1}{2}\delta^2_xf|_y\delta x^2
    \end{equation}

\end{theorem}

\begin{theorem}[Second Order Necessary Condition; $(\mathbb{R},d)$]
    
    If $x^* \in \Omega^\circ \subseteq \mathbb{R}$ is a local minimum for $f\in C^2_b(\Omega;\mathbb{R})$, then

    \begin{equation}
        \pdv[2]{f}{x}\big|_{x^*} \ge 0
    \end{equation}

\end{theorem}

\begin{theorem}[Second Order Sufficient Condition; $(\mathbb{R},d)$]
    
    Prerequisites
    \begin{itemize}
        \item Let $x^* \in \Omega^\circ \subseteq \mathbb{R}$
        \item Assume $f \in C^2_b(\Omega;\mathbb{R})$
        \item If $\pdv{f}{x}|_{x^*} = 0$ and $\pdv[2]{f}{x}|_{x^*} > 0$ hold
    \end{itemize}

    Then $x^*$ is a strict local minimum for $f$.
\end{theorem}

\begin{definition}[Unit Sphere]
    
    Defining
    \begin{itemize}
        \item $S^n(x,d)\subset \mathbb{R}^{n+1}$ for $n \in \mathbb{N}$: unit sphere
        \item $x\in \mathbb{R}^{n+1}$: a point
        \item $d$: a metric on $\mathbb{R}^{n+1}$
    \end{itemize}

    Then the complete definition is $S^n(x,d) \equiv \{y\in \mathbb{R}^{n+1} : d(x,y) = 1\}$

    \begin{remark}
        If $S^n$ is denoted without $x$ and $d$, then it can be assumed that $x=0$ and $d$ is the standard $l_2$ metric.
    \end{remark}
    
\end{definition}

\begin{theorem}[First Order Necessary Condition; $(\mathbb{R}^n,d)$]
    
    Given
    \begin{itemize}
        \item $f\in C^1(\Omega;\mathbb{R})$
        \item $x^* \in \Omega^\circ \subseteq \mathbb{R}^n$
    \end{itemize}

    If $x^*$ is a local minimum for $f$, then
    
    \begin{equation*}
        \nabla f\big|_{x^*} \equiv \bigg(\pdv{f}{x_1}\bigg|,\cdots,\pdv{f}{x_n}\bigg|_{x^*}\bigg) = 0 \in \mathbb{R}^n
    \end{equation*}
    
\end{theorem}

\begin{definition}[Hessian Matrix]
    
    Given
    \begin{itemize}
        \item $f\in C^2(\mathbb{R}^n;\mathbb{R})$
        \item $\mathbb{S}(n;\mathbb{R}) = \mathbb{S}(n)$: an $n\times n$ symmetric matrix
    \end{itemize}

    The Hessian matrix of $f$ evaluated at $x \in \mathbb{R}$ is

    \begin{equation}
        \nabla^{\otimes 2}_x f = \nabla^2_x = \begin{bmatrix}
            \pdv[2]{f}{x_1} & \cdots & \pdv{f}{x_1x_n} \\
            \vdots & \ddots & \vdots \\
            \pdv{f}{x_nx_1} & \cdots & \pdv[2]{f}{x_n}
        \end{bmatrix}
    \end{equation}

    \begin{remark}
        Every $A\in \mathbb{S}(n)$ can be expressed as $A = U\Lambda U^T$ where the columns of $U$ are the eigenvectors of $A$ and the diagonal of $\Lambda$ is the eigenvalues of $A$.
    \end{remark}
\end{definition}